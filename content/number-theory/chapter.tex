\chapter{Number theory}

\section{Modular arithmetic}
	\kactlimport{ModMulLL.java}
	\kactlimport{ModInverse.java}
	\kactlimport{ModLog.h}
	\kactlimport{ModSum.h}
	\kactlimport{ModSqrt.h}

\section{Primality}
    \kactlimport{SieveOfEratosthenes.java}
	\kactlimport{SieveOfErastosthenesFast.java}
	\kactlimport{MillerRabin.java}
	\kactlimport{Factor.h}

\section{Divisibility}
	\kactlimport{Euclid.java}
	\kactlimport{CRT.java}

	\subsection{Identidad de Bezut}
	Para $a \neq $, $b \neq 0$, entonces $d=gcd(a,b)$ es el menor entero positivo para el que hay soluciones enteras de
	$$ax+by=d$$
	SI $(x,y)$ es una solucion, entonces todas las soluciones enteras vienen dadas por:
	$$\left(x+\frac{kb}{\gcd(a,b)}, y-\frac{ka}{\gcd(a,b)}\right), \quad k\in\mathbb{Z}$$

	\kactlimport{phiFunction.java}
\section{Fractions}
	\kactlimport{ContinuedFractions.h}
	\kactlimport{FracBinarySearch.h}

\section{Ternas Pitagoricas}
 Las ternas pitagoricas son generadas de forma unica por
 \[ a=k\cdot (m^{2}-n^{2}),\ \,b=k\cdot (2mn),\ \,c=k\cdot (m^{2}+n^{2}), \]
 con $m > n > 0$, $k > 0$, $m \bot n$, ni $m$ o $n$ par.

\section{Primos}
	$p=962592769$ es $2^{21} \mid p-1$, puede ser util. Para hashing
	usar 970592641 (31-bit number), 31443539979727 (45-bit), 3006703054056749
	(52-bit). Hay  78498 primos menores que 1\,000\,000.

	Raices primitivas existen modulo cualquier potencia prima $p^a$, excepto para $p = 2, a > 2$, y hay muchos $\phi(\phi(p^a))$ .
	Para $p = 2, a > 2$, el grupo $\mathbb Z_{2^a}^\times$ es isomorfo a $\mathbb Z_2 \times \mathbb Z_{2^{a-2}}$.

\section{Estimates}
	$\sum_{d|n} d = O(n \log \log n)$.

	EL numero de divisores de $n$ es cercano a 100 para $n < 5e4$, 500 para $n < 1e7$, 2000 para $n < 1e10$, 200\,000 para $n < 1e19$.

\section{Mobius Function}
\[
	\mu(n) = \begin{cases} 0 & n \textrm{ no tiene factores primos repetidos}\\ 1 & n \textrm{ tiene un numero par de factores primos}\\ -1 & n \textrm{ tiene un numero impar de factores primos}\\\end{cases}
\]
  Mobius Inversion:

  \[ g(n) = \sum_{d|n} f(d) \Leftrightarrow f(n) = \sum_{d|n} \mu(d)g(n/d) \]
  Otras formulas utiles:\\
    La suma sobre todos los divisores positivos de n de la función de Mobius es cero excepto cuando n = 1.
  $ \sum_{d | n} \mu(d) = [ n = 1] $ (Muy util)

  $ g(n) = \sum_{n|d} f(d) \Leftrightarrow f(n) = \sum_{n|d} \mu(d/n)g(d)$

 $ g(n) = \sum_{1 \leq m \leq n} f(\left\lfloor\frac{n}{m}\right \rfloor ) \Leftrightarrow f(n) = \sum_{1\leq m\leq n} \mu(m)g(\left\lfloor\frac{n}{m}\right\rfloor)$
